\documentclass[9pt]{ltjsarticle}
\DeclareSymbolFont{bbold}{U}{bbold}{m}{n}
\DeclareSymbolFontAlphabet{\mathbbold}{bbold}
\newcommand{\bbold}{\mathbbold}
\usepackage{xcolor}
\usepackage{amsmath,amsfonts,amssymb}
\usepackage{enumitem}
\usepackage{ashiato45}
%\usepackage{okumacro}
\def\MARU#1{\textcircled{\scriptsize #1}}
\usepackage{graphicx}
\usepackage{ulem}
\usepackage{framed}
\usepackage{algorithm}
\usepackage{algorithmic}
\usepackage{here}
%\usepackage[twoside]{geometry}
\usepackage{mytheorems}
\usepackage{tikz}
\usepackage{ascmac}
\usepackage{stmaryrd}
\usepackage{txfonts}
\usetikzlibrary{cd}
\title{先端データ解析論(杉山将先生・本多淳也先生)\\第3回レポート}
\author{ashiato45}
\date{2017年5月1日}

\renewcommand{\bf}{\mathbf}
\newcommand{\nemui}{Y=眠}
\newcommand{\nemukunai}{Y=非眠}
\newcommand{\suki}{X=好}
\newcommand{\kirai}{X=嫌}
\newcommand{\argmin}{\mathop{\mathrm{argmin}}}
\newcommand{\argmax}{\mathop{\mathrm{argmax}}}

\begin{document}
\maketitle

\section*{宿題1}
\begin{align}
 T_+(z) &= z^2/2 + (\lambda-u-\theta)z + (u\theta + \theta^2/2),\\
 T_-(z)&= z^2/2 + (-\lambda-u-\theta)z + (u\theta + \theta^2/2)
\end{align}
と定義する。両方とも最高次が正な2次関数であり、$T_+'(z)=z+(\lambda-u-\theta),
T_-'(z)=z+(-\lambda-u-\theta)$なので、
\begin{align}
 \argmin_{z\ge 0} T(z) &= \max\left(\argmin_z T_+(z), 0\right) = \max(\theta+u-\lambda, 0),\\
 \argmin_{z\le 0}T(z)& = \min\left(\argmin_z T_-(z), 0\right) = \min(\lambda+u+\theta, 0).
\end{align}
よって、
\begin{align}
 \argmin_z T(z) = \argmin_{z\ge 0} T(z) + \argmax_{z\le 0} T(z) = \max(\theta+u-\lambda, 0) + \min(\lambda+u+\theta, 0).
\end{align}

\section*{宿題2}
\begin{align}
 \Phi = 
\begin{pmatrix}
 K(x_1,x_1) & \ldots & K(x_1,x_n)\\
 \vdots&\ddots & \vdots\\
 K(x_n,x_1)&\ldots & K(x_n,x_n)
\end{pmatrix}
\end{align}
として、以下の反復
\begin{align}
 \theta & \leftarrow (\Phi^\top\Phi + I)\inv (\Phi^\top y + z-u),\\
 z& \leftarrow \max(0,\theta+u-\lambda) - \max(0,-\theta-u-\lambda),\\
 u& \leftarrow u+\theta-z
\end{align}
を行えばよい。今回は停止条件として「更新の幅が十分小さくなったら」という
ものを用いた。計算に用いたプログラムは付録に記す。パラメタは、$h=0.3,
\lambda=0.2$とした。

結果、図1を得た。50個のパラメタのうち、38個は$10^{-8}$以下であり、
ほぼ0と見做せた。誤差のノルムは$1.14$であった。
\easypicture{figure_1.png}




\section*{付録}
\tiny
\begin{verbatim}
import sys
import numpy as np
import matplotlib.pyplot as plt

np.random.seed(42)

h = float(sys.argv[1])
lam = float(sys.argv[2])

datanum = 50
datasamples = np.linspace(-3, 3, datanum)
datasamples = datasamples.reshape(len(datasamples), 1)
y = np.sin(np.pi*datasamples)/(np.pi*datasamples) + 0.1*datasamples
y += np.random.randn(datanum, 1)*0.2 # normal distribution
# print(y)

def gk(x, c):
    return np.exp(-(x-c)**2/(2*(h**2)))

phi = np.fromfunction(lambda i, j: gk(datasamples[i, 0], datasamples[j, 0]), (datanum, datanum), dtype=int)

theta = np.zeros((datanum, 1))
z = np.zeros((datanum, 1))
u = np.zeros((datanum, 1))

while True:
    theta2 = np.linalg.solve(np.dot(phi.transpose(), phi) + np.eye(datanum, dtype=int), np.dot(phi.transpose(), y) + z - u)
    z2 = np.maximum(0, theta2+u-lam) + np.minimum(0, theta2+u+lam)
    u2 = u + theta2 - z2
    dt = theta-theta2
    dz = z-z2
    du = u-u2
    theta = theta2
    z = z2
    u = u2
    if np.all(abs(dt) < 1e-9) and np.all(abs(dz) < 1e-9) and np.all(abs(du) < 1e-9):
        break
        

graphnum = 5000
graphsamples = np.linspace(-3, 3, graphnum)
graphsamples = graphsamples.reshape(graphnum, 1)
calckernel = lambda x: np.dot(theta.transpose(), np.fromfunction(lambda i, j:gk(x, datasamples[i, 0]), (datanum, 1), dtype=int))[0, 0]
km = np.fromfunction(lambda i,j: gk(graphsamples[i, 0], datasamples[j, 0]), (graphnum, datanum), dtype=int)
graph = np.dot(km, theta)

em = np.fromfunction(lambda i,j: gk(datasamples[i, 0], datasamples[j, 0]), (datanum, datanum), dtype=int)
eout = np.dot(em, theta)
er = eout - y

print(theta)
print(np.sum(abs(theta) < 1e-8))
print(np.linalg.norm(er))

plt.axis([-2.8, 2.8, -1, 1.5])    
plt.plot(datasamples, y, 'o')
plt.plot(graphsamples, graph, '-')
# plt.plot(datasamples, eout, 'o')
plt.show()
\end{verbatim}

\end{document}
