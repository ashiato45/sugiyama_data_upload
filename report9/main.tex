\documentclass[9pt]{ltjsarticle}
\DeclareSymbolFont{bbold}{U}{bbold}{m}{n}
\DeclareSymbolFontAlphabet{\mathbbold}{bbold}
\newcommand{\bbold}{\mathbbold}
\usepackage{xcolor}
\usepackage{amsmath,amsfonts,amssymb}
\usepackage{enumitem}
\usepackage{ashiato45}
%\usepackage{okumacro}
\def\MARU#1{\textcircled{\scriptsize #1}}
\usepackage{graphicx}
\usepackage{ulem}
\usepackage{framed}
\usepackage{algorithm}
\usepackage{algorithmic}
\usepackage{here}
%\usepackage[twoside]{geometry}
\usepackage{mytheorems}
\usepackage{tikz}
% usepackage{ascmac}
% \usepackage{stmaryrd}
% \usepackage{txfonts}
% \usetikzlibrary{cd}
\title{先端データ解析論(杉山将先生・本多淳也先生)\\第9回レポート}
\author{ashiato45}
\date{2017年6月13日}

\renewcommand{\bf}{\mathbf}
\newcommand{\nemui}{Y=眠}
\newcommand{\nemukunai}{Y=非眠}
\newcommand{\suki}{X=好}
\newcommand{\kirai}{X=嫌}
\newcommand{\argmin}{\mathop{\mathrm{argmin}}}
\newcommand{\argmax}{\mathop{\mathrm{argmax}}}
\newcommand{\test}{\mathrm{test}}
\newcommand{\train}{\mathrm{train}}
\newcommand{\kitai}{\bbold{E}}

\begin{document}
\maketitle

\section*{宿題1}
C++とEigenによる実装(表示にはPythonを用いた)は付録1にある。結果、図1を得た。

\easypicture{figure_1.pdf}

\section*{宿題2}
\begin{align}
J(\pi)
&=
2\kitai_{x'\sim p_\test, x\sim q_\pi} \norm{x'-x} - \kitai_{x,\tilde x\sim q_\pi}\norm{x-\tilde x} + \text{Const.}
\end{align}
が成り立つ。それぞれの項を計算する。
\begin{align}
\kitai_{x'\sim p_\test, x\sim q_\pi} \norm{x'-x}
&=
\pi \kitai_{x'\sim p_\test, x\sim p_\train(x\mid y=+1)} \norm{x'-x} + (1-\pi)\kitai_{x'\sim p_\test, x\sim p_\train(x\mid y=-1)}\norm{x'-x}\\
&=
\pi b_{+1} - \pi b_{-1} + \text{Const.}
\end{align}

\begin{align}
\kitai_{x,\tilde x\sim q_\pi}\norm{x-\tilde x}
&=
\pi^2\kitai_{x\sim p_\train(x\mid y=+1),\tilde x\sim p_\train(x\mid y=+1)}\norm{x-\tilde x}\\
&\quad+ 2\pi(1-\pi)\kitai_{x\sim p_\train(x\mid y=+1),\tilde x\sim p_\train(x\mid y=-1)}\norm{x-\tilde x}\\
&\quad+ (1-\pi)^2\kitai_{x\sim p_\train(x\mid y=-1),\tilde x\sim p_\train(x\mid y=-1)}\norm{x-\tilde x}\\
&= \pi^2 A_{+1,+1} + 2\pi A_{+1,-1} -2\pi^2 A_{+1,-1} -2\pi A_{-1,-1} +\pi^2 A_{-1,-1} + \text{Const.}
\end{align}
これらをまとめて、
\begin{align}
J(\pi) = (2A_{+1,-1} - A_{+1,+1} - A_{-1,-1})\pi^2
-2(A_{+1,-1} - A_{-1,-1}- b_{+1}+ b_{-1})\pi + \text{Const.}
\end{align}
を得る。

\section*{宿題3}
C++とEigenによる実装(表示にはPythonを用いた)は付録2にある。結果、図2,3を得た。
図2は学習データとそれに対する重みなし最小二乗法の結果であり、
図3はテストデータとそれに対するクラス比重み付き最小二乗法の結果である。

\easypicture{figure_2.pdf}\easypicture{figure_3.pdf}

\section*{付録1}
\subsection*{学習プログラム}
\tiny
\begin{verbatim}
#include <iostream>
#include <Eigen/Dense>
#include <cmath>
#include <random>
#include <string>
#include <fstream>
#define M_PI 3.1416

#define print(var)  \
  std::cout<<#var"= "<<std::endl<<(var)<<std::endl

using Eigen::MatrixXd;

void save_mat(std::string& name, const MatrixXd* mat){
  std::ofstream file(name);
  file << *mat << std::endl;
}

float k(const MatrixXd& x, const MatrixXd& c){
  float hh = std::pow(2*1, 2); 
  auto d = (x - c).norm();
  return std::exp(-d*d/hh);
}

int main()
{
  std::mt19937 engine;
  std::normal_distribution<> dist(0, 1);

  MatrixXd a(100, 1);
  for(int i=0; i < 100; i++){
    a(i) = M_PI*i/100;
  }
  MatrixXd x(2, 200);
  for(int i=0; i < 100; i++){
      x(0, i) = -10*(std::cos(a(i)) + 0.5) + dist(engine);
      x(1, i) = 10*std::sin(a(i)) + dist(engine);
      x(0, i + 100) = -10*(std::cos(a(i)) - 0.5) + dist(engine);
      x(1, i + 100) = -10*std::sin(a(i)) + dist(engine);
  }
  MatrixXd y(1, 200);
  y(0) = -1;
  y(199) = 1;
  MatrixXd yy(1, 2);
  yy(0) = -1;
  yy(1) = 1;


  /* Learn */
  MatrixXd phi(200, 200);
  for(int i=0; i < 200; i++){
    for(int j=0; j < 200; j++){
      phi(i, j) =  k(x.col(i), x.col(j));
    }
  }
  MatrixXd phit(2, 200);
  for(int j=0; j < 200; j++){
    phit(0, j) = k(x.col(0), x.col(j));
    phit(1, j) = k(x.col(199), x.col(j));
  }
  MatrixXd w = phi;
  MatrixXd d(200, 200);
  for(int i=0; i < 200; i++){
    d(i, i) = w.col(i).sum();
  }
  MatrixXd l = d - w;

  MatrixXd A = phit.transpose()*phit + MatrixXd::Identity(200, 200) + 2*phi.transpose()*l*phi;
  MatrixXd b = phit.transpose()*yy.transpose();
  MatrixXd theta = A.colPivHouseholderQr().solve(b);

  /* Output */
  MatrixXd out(100, 100);
  for(int i=0; i < 100; i++){
    for(int j=0; j < 100; j++){
      float mx = -20*(1.0 - i/100.0) + 20*(i/100.0);
      float my = -20*(1.0 - j/100.0) + 20*(j/100.0);
      for(int kk = 0; kk < 200; kk++){
        MatrixXd xx(2, 1);
        xx << mx, my;
        out(j, i) += theta(kk)*k(x.col(kk), xx);
      }
    }
  }

  std::ofstream file("x");
  file << x << std::endl;
  std::ofstream fileout("out");
  fileout << out << std::endl;
  //std::cout << m << std::endl;
}
\end{verbatim}
\normalsize
\subsection*{表示プログラム}
\tiny
\begin{verbatim}
import numpy as np
import matplotlib.pyplot as plt

x = np.loadtxt("x")

v2 = np.loadtxt("out")

plt.axis([-20, 20, -20, 20])
lsp = np.linspace(-20, 20, 100)
plt.contourf(lsp, lsp, np.sign(v2))
plt.plot(x[0, :], x[1, :], 'ko')   
plt.plot(x[0, 0], x[1, 0], 'bo')   
plt.plot(x[0, 199], x[1, 199], 'ro')   
plt.show()
\end{verbatim}

\normalsize

\section*{付録2}

\subsection*{学習プログラム}
\tiny
\begin{verbatim}
#include <iostream>
#include <Eigen/Dense>
#include <cmath>
#include <random>
#include <string>
#include <fstream>
#include <algorithm>
#define M_PI 3.1416

#define print(var)  \
  std::cout<<#var"= "<<std::endl<<(var)<<std::endl

using Eigen::MatrixXd;

float gk(const MatrixXd& x, const MatrixXd& c){
  float hh = std::pow(2*1, 2); 
  auto d = (x - c).norm();
  return std::exp(-d*d/hh);
}

int main()
{
  std::mt19937 engine;
  std::normal_distribution<> dist(0, 1);

  /* Making data */
  MatrixXd px(3, 100);
  MatrixXd py(1, 100);
  for(int i=0; i < 90; i++){
    px(0, i) = dist(engine) - 2;
    px(1, i) = 2*dist(engine);
    px(2, i) = 1;
    py(i) = -1;
  }
  for(int i=90; i < 100; i++){
    px(0, i) = dist(engine) + 2;
    px(1, i) = 2*dist(engine);
    px(2, i) = 1;
    py(i) = 1;
  }
  MatrixXd tx(3, 100);
  for(int i=0; i < 10; i++){
    tx(0, i) = dist(engine) - 2;
    tx(1, i) = 2*dist(engine);
    tx(2, i) = 1;
  }
  for(int i=10; i < 100; i++){
    tx(0, i) = dist(engine) + 2;
    tx(1, i) = 2*dist(engine);
    tx(2, i) = 1;
  }

  /* Learn */
  float hat_a_11 = 0;
  int nn_11 = 0;
  float hat_a_12 = 0;
  int nn_12 = 0;
  float hat_a_22 = 0;
  int nn_22 = 0;
  for(int i=0; i < 100; i++){
    for(int j=0; j < 100; j++){
      if(py(i) == -1 && py(j) == -1){
        hat_a_11 += (px.col(i) - px.col(j)).norm();
        nn_11++;
      }else if(py(i) == -1 && py(j) == 1){
        hat_a_12 += (px.col(i) - px.col(j)).norm();
        nn_12++;
      }else if(py(i) == 1 && py(j) == 1){
        hat_a_22 += (px.col(i) - px.col(j)).norm();
        nn_22++;
      }
    }
  }
  hat_a_11 /= (float)nn_11;
  hat_a_12 /= (float)nn_12;
  hat_a_22 /= (float)nn_22;
  float hat_b_1 = 0;
  int ndn_1 = 0;
  float hat_b_2 = 0;
  int ndn_2 = 0;
  for(int i=0; i < 100; i++){
    for(int j=0; j < 100; j++){
      if(py(j) == -1){
        hat_b_1 += (tx.col(i) - px.col(j)).norm();
        ndn_1++;
      }else if(py(j) == 1){
        hat_b_2 += (tx.col(i) - px.col(j)).norm();
        ndn_2++;
      }
    }
  }
  hat_b_1 /= (float)ndn_1;
  hat_b_2 /= (float)ndn_2;

  float til_pi = (hat_a_12 - hat_a_11 - hat_b_2 + hat_b_1)/(2*hat_a_12 - hat_a_22 - hat_a_11);
  float hat_pi = std::min(1.0f, std::max(0.0f, til_pi));

  /* Learn */
  std::cout << hat_pi << std::endl;
  MatrixXd til_w(100, 100);
  for(int i=0; i < 100; i++){
    if(py(i) == 1){
      til_w(i, i) = hat_pi;
    }else{
      til_w(i, i) = 1.0 - hat_pi;
    }
  }
  MatrixXd phi = px.transpose();
  MatrixXd theta = (phi.transpose()*til_w*phi).colPivHouseholderQr().solve(phi.transpose()*til_w*py.transpose());
  MatrixXd theta2 = (phi.transpose()*phi).colPivHouseholderQr().solve(phi.transpose()*py.transpose());


  /* Output */
  std::ofstream pxfile("px");
  pxfile << px << std::endl;
  std::ofstream txfile("tx");
  txfile << tx << std::endl;
  std::ofstream thetafile("theta");
  thetafile << theta << std::endl;
  std::ofstream theta2file("theta2");
  theta2file << theta2 << std::endl;
  //std::cout << m << std::endl;
}
\end{verbatim}
\normalsize
\subsection*{表示プログラム}
\tiny
\begin{verbatim}
import numpy as np
import matplotlib.pyplot as plt

theta = np.loadtxt("theta")
theta2 = np.loadtxt("theta2")

tx = np.loadtxt("tx")
px = np.loadtxt("px")

plt.axis([-5, 5, -10, 10])
p = np.linspace(-5, 5, 10)
plt.plot(p, -(theta2[2] + p*theta2[0])/theta2[1], 'g-')   
plt.plot(px[0, 0:90], px[1, 0:90], 'bo')   
plt.plot(px[0, 90:], px[1, 90:], 'ro')   
plt.show()

plt.axis([-5, 5, -10, 10])
p = np.linspace(-5, 5, 10)
plt.plot(p, -(theta[2] + p*theta[0])/theta[1], 'g-')   
plt.plot(tx[0, 0:10], tx[1, 0:10], 'bo')   
plt.plot(tx[0, 10:], tx[1, 10:], 'ro')   
plt.show()
\end{verbatim}

\normalsize

\end{document}
