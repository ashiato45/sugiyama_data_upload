\documentclass[9pt]{ltjsarticle}
\DeclareSymbolFont{bbold}{U}{bbold}{m}{n}
\DeclareSymbolFontAlphabet{\mathbbold}{bbold}
\newcommand{\bbold}{\mathbbold}
\usepackage{xcolor}
\usepackage{amsmath,amsfonts,amssymb}
\usepackage{enumitem}
\usepackage{ashiato45}
%\usepackage{okumacro}
\def\MARU#1{\textcircled{\scriptsize #1}}
\usepackage{graphicx}
\usepackage{ulem}
\usepackage{framed}
\usepackage{algorithm}
\usepackage{algorithmic}
\usepackage{here}
%\usepackage[twoside]{geometry}
\usepackage{mytheorems}
\usepackage{tikz}
\usepackage{ascmac}
\usepackage{stmaryrd}
\usepackage{txfonts}
\usetikzlibrary{cd}
\title{先端データ解析論(杉山将先生・本多淳也先生)\\第4回レポート}
\author{ashiato45}
\date{2017年5月9日}

\renewcommand{\bf}{\mathbf}
\newcommand{\nemui}{Y=眠}
\newcommand{\nemukunai}{Y=非眠}
\newcommand{\suki}{X=好}
\newcommand{\kirai}{X=嫌}
\newcommand{\argmin}{\mathop{\mathrm{argmin}}}
\newcommand{\argmax}{\mathop{\mathrm{argmax}}}

\begin{document}
\maketitle

\section*{宿題1}
\begin{align}
 e(\theta) = \frac{1}{2}\sum_{i=1}^n \tilde w_i 
\left(\sum_{j=1}^b \theta_j \phi_j(x_i) - y_i\right)^2
\end{align}
とおく。$e$は下に凸な二次関数なので、微分が0の点を求めれば最小点を求めることができ
る。

$J=1,\dots,b$として、
\begin{align}
 \frac{\pd}{\pd \theta_J}e(\theta) = 
\sum_{i=1}^n \tilde w_i \left(\sum_{j=1}^b \theta_j\phi_j(x_i)-y_i\right)\phi_J(x_i)
\end{align}
となるので、これが0になるときを考えると、次の式を得る:
\begin{align}
 \sum_{i=1}^n \tilde w_i y_i \phi_J(x_i) = \sum_{i=1}^n \tilde w_i \sum_{j=1}^b \theta_j \phi_j(x_i)\phi_J(x_i).
\end{align}
これが各$J$について成立するので、
\begin{align}
 \Phi^\top \tilde W y = \Phi^\top \tilde W \Phi \theta.
\end{align}
よって、$e$を最小化する$\theta$である$\hat \theta$は、
\begin{align}
 \hat \theta = (\Phi^\top \tilde W \Phi \theta)\inv \Phi^\top \tilde W y.
\end{align}


\section*{宿題2}
対称性より、未知数$a,b$を用いて$\tilde \rho(r)=ar^2+b$とおいてよい。
$\tilde(r)$は$(\tilde r, \rho(\tilde r))$で$\rho(r)$に接するので、
$\tilde \rho'(\tilde r)=\rho'(\tilde r)$と$\tilde \rho(\tilde
r)=\rho(\tilde r)$とを満たす。これは、
$2a\tilde r = \rho'(\tilde r)$と$a\tilde r^2+b=\rho(\tilde r)$を得るので、
これを解いて
\begin{align}
 \tilde \rho(r) = \frac{\rho'(\tilde r)}{2\tilde r}r^2 + b
\end{align}
となる。

\section*{宿題3}
Python実装は付録にある。$\eta = 1$とした。
結果、図1を得た。
\easypicture{figure_1.png}




\section*{付録}
\tiny
\begin{verbatim}
import sys
import numpy as np
import matplotlib.pyplot as plt

np.random.seed(42)

datanum = 10
datasamples = np.linspace(-3, 3, datanum)
datasamples = datasamples.reshape(len(datasamples), 1)
y = datasamples + 0.2*np.random.randn(datanum, 1)
y[datanum - 1] = -4
y[datanum - 2] = -4
y[3] = -4

eta = 1
theta = np.zeros((2, 1))
phi = np.ones((datanum, 2))
phi[:, 1] = datasamples.reshape(len(datasamples))
for i in range(10000):
    r = np.abs(np.dot(phi, theta) - y)
    w = np.zeros((datanum, 1))
    w[r <= eta] = ((1-r**2/eta**2)**2)[r <= eta]
    W = np.diag(w.reshape(len(w)))
    A = np.dot(np.dot(phi.transpose(), W), phi)
    b = np.dot(np.dot(phi.transpose(), W), y)
    theta2 = np.linalg.solve(A, b)
    if np.linalg.norm(theta2 - theta) < 0.001:
        break
    theta = theta2

graphnum = 5000
graphsamples = np.linspace(-3, 3, graphnum)
graphsamples = graphsamples.reshape(graphnum, 1)
graph = theta[0, 0] + theta[1, 0]*graphsamples

plt.axis([-3.2, 3.2, -4.2, 4.2])    
plt.plot(datasamples, y, 'o')
plt.plot(graphsamples, graph, '-')
plt.show()
\end{verbatim}

\end{document}
