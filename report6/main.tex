\documentclass[9pt]{ltjsarticle}
\DeclareSymbolFont{bbold}{U}{bbold}{m}{n}
\DeclareSymbolFontAlphabet{\mathbbold}{bbold}
\newcommand{\bbold}{\mathbbold}
\usepackage{xcolor}
\usepackage{amsmath,amsfonts,amssymb}
\usepackage{enumitem}
\usepackage{ashiato45}
%\usepackage{okumacro}
\def\MARU#1{\textcircled{\scriptsize #1}}
\usepackage{graphicx}
\usepackage{ulem}
\usepackage{framed}
\usepackage{algorithm}
\usepackage{algorithmic}
\usepackage{here}
%\usepackage[twoside]{geometry}
\usepackage{mytheorems}
\usepackage{tikz}
% usepackage{ascmac}
% \usepackage{stmaryrd}
% \usepackage{txfonts}
% \usetikzlibrary{cd}
\title{先端データ解析論(杉山将先生・本多淳也先生)\\第7回レポート}
\author{ashiato45}
\date{2017年5月29日}

\renewcommand{\bf}{\mathbf}
\newcommand{\nemui}{Y=眠}
\newcommand{\nemukunai}{Y=非眠}
\newcommand{\suki}{X=好}
\newcommand{\kirai}{X=嫌}
\newcommand{\argmin}{\mathop{\mathrm{argmin}}}
\newcommand{\argmax}{\mathop{\mathrm{argmax}}}

\begin{document}
\maketitle

\section*{宿題1}
\begin{enumerate}
 \item $\alpha_i=0$とする。$\alpha_i+\beta_i=C$なので$\beta_i=C>0$とな
       る。よって、$\beta_i\xi_i=0$より、$\xi_i=0$とならなければならない。
       $y_iw^\top x_i-1+\xi_i\ge 0$にこれを代入し、$y_iw^\top x_i \ge 1$
       を得る。
 \item $0<\alpha_i<C$とする。$\alpha_i+\beta_i=C$と$\alpha_i<C$より$\beta_i > 0$であ
       る。よって、$\xi_i=0$となる。$0<\alpha$より$y_iw^\top
       x_i-1+\xi_i=0$となる。ここに$\xi_i=0$を代入し、$y_iw^\top x_i =
       1$を得る。
 \item $\alpha_i=C$とする。$\alpha_i>0$なので$y_iw^\top x_i - 1+\xi_i
       = 0$となる。$y_iw^\top x_i = 1-\xi_i\le 1$となる。
 \item $y_iw^\top x_i > 1$とする。$y_iw^\top x_i-1+\xi_i > \xi_i \ge 0$
       である。これと$\alpha_i(y_iw^\top x_i-1+\xi_i)=0$より、
       $\alpha_i=0$を得る。
 \item $y_iw^\top x_i < 1$とする。$\alpha_i\ge 0$であることと2の対偶よ
       り、$\alpha_i=0$であるか$\alpha_i\ge C$のどちらかである。さらに
       1の対偶より$\alpha_i\neq 0$がわかるので、$\alpha_i \ge C$である。
       $\beta_i\ge 0$と$\alpha_i+\beta_i=C$より$\alpha_i \le C$がわかる
       ので、$\alpha_i = C$である。
\end{enumerate}



\section*{宿題2}
以下の最小化を行なった:
\begin{align}
 \min_{w, b} \sum_{i=1}^{200} \max(0, 1-y_i f_{w, b}(x_i)) + \frac{\lambda}{2}(w_1, w_2, b)(w_1, w_2, b)^\top
\end{align}
ここで、$\lambda=1$とした。最適化には劣勾配法を用い、終了条件は「最適化
するベクトルがあまり動かなくなったら」とした。すると、図1を得た。
\easypicture{figure_1.pdf}

Pythonでの実装は付録にある。

\section*{付録}
\tiny
\begin{verbatim}
""" An assignment of ADA 6"""

import numpy as np
import numpy.matlib
import matplotlib.pyplot as plt

np.random.seed(42)

H = 1
L = 1
EPS = 10e-6

N = 200
X = np.hstack([np.random.randn(1, N//2) - 5, np.random.randn(1, N//2) + 5])
# add third line to simplify the calculation of f.
X = np.vstack([X, np.random.randn(1, N), np.ones((1, N))])
X = X.transpose()  # Make two groups
Y = np.vstack([np.ones((N//2, 1)), -np.ones((N//2, 1))])
X[0:3, 1] -= 5 # Make exceptional data
X[N//2:N//2+3, 1] += 5
Y[0:3] = -1
Y[N//2:N//2+3] = 1

def f(theta_, x_):
    return np.dot(theta_.transpose(), x_)


theta = np.ones((3, 1))
while True:
    d = -X*np.matlib.repmat(Y, 1, 3)  # Nx3
    ind = 1 - Y*X.dot(theta)
    d[ind <= 0] = 0
    d = np.sum(d, axis=0) #1x3
    d = d.reshape(3, 1) #3x1
    d += L*theta
    theta2 = theta - EPS*d
    err = np.linalg.norm(theta2 - theta)
    print(err)
    theta = theta2
    if err < 10e-5:
        break


print(theta)        


plt.axis([-10, 10, -10, 10])
plt.plot(X[(Y == 1).flatten(), 0], X[(Y == 1).flatten(), 1], 'bo')
plt.plot(X[(Y == -1).flatten(), 0], X[(Y == -1).flatten(), 1], 'rx')
plt.plot(np.array([-10, 10]), -(theta[2] + np.array([-10, 10])*theta[0])/theta[1], 'k-')
plt.show()
\end{verbatim}

\end{document}
