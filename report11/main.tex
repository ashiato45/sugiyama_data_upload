\documentclass[9pt]{ltjsarticle}
\DeclareSymbolFont{bbold}{U}{bbold}{m}{n}
\DeclareSymbolFontAlphabet{\mathbbold}{bbold}
\newcommand{\bbold}{\mathbbold}
\usepackage{xcolor}
\usepackage{amsmath,amsfonts,amssymb}
\usepackage{enumitem}
\usepackage{ashiato45}
%\usepackage{okumacro}
\def\MARU#1{\textcircled{\scriptsize #1}}
\usepackage{graphicx}
\usepackage{ulem}
\usepackage{framed}
\usepackage{algorithm}
\usepackage{algorithmic}
\usepackage{here}
%\usepackage[twoside]{geometry}
\usepackage{mytheorems}
\usepackage{tikz}
% usepackage{ascmac}
% \usepackage{stmaryrd}
% \usepackage{txfonts}
% \usetikzlibrary{cd}
\title{先端データ解析論(杉山将先生・本多淳也先生)\\第11回レポート}
\author{ashiato45}
\date{2017年7月4日}

\renewcommand{\bf}{\mathbf}
\newcommand{\nemui}{Y=眠}
\newcommand{\nemukunai}{Y=非眠}
\newcommand{\suki}{X=好}
\newcommand{\kirai}{X=嫌}
\newcommand{\argmin}{\mathop{\mathrm{argmin}}}
\newcommand{\argmax}{\mathop{\mathrm{argmax}}}
\newcommand{\test}{\mathrm{test}}
\newcommand{\train}{\mathrm{train}}
\newcommand{\kitai}{\bbold{E}}
\newcommand{\tr}{\mathrm{tr}}

\begin{document}
\maketitle

\section*{宿題1}
C++とEigenによる実装(表示にはPythonを用いた)は付録1にある。結果、図1と図2を得た。
  \easypicture{figure_1.pdf}
  \easypicture{figure_2.pdf}

\section*{宿題2}
前半を示す。
\begin{align}
S\ovparen{w} 
&=
\sum_{y=1}^c \sum_{i\colon y_i = y} (x_i - \mu_y)(x_i-\mu_y)^\top\\
&=
\sum_y \sum_i \left(x_ix_i^\top - x_i \sum_{j\colon y_j=y}\frac{1}{n_y}x_j^\top - \sum_{k\colon y_k=y}\frac{1}{n_y}x_kx_i^\top + \frac{1}{n_y^2}\sum_j\sum_k x_jx_k^\top \right) \\
&=
\sum_y \left(\sum_i x_ix_i^\top -\frac{2}{n_y}\sum_{i,j}x_ix_j^\top + \frac{1}{n_y^2}\sum_{i,j,k}x_jx_k^\top \right)\\
&=
\sum_y \left(\sum_i x_ix_i^\top - \frac{1}{n_y}\sum_{i,j}x_ix_j^\top\right).
\end{align}
一方、
\begin{align}
\frac{1}{2}\sum_{i,j=1}^n Q_{i,j}\ovparen{w}(x_i-x_j)(x_i-x_j)^\top
&=
\frac{1}{2}\sum_{y=1}^c\sum_{i\colon y_i=y}\sum_{z=1}^c \sum_{j\colon y_j=z} Q_{i,j}\ovparen{w}(x_i-x_j)(x_i-x_j)^\top\\
&=
\frac{1}{2}\sum_{y=1}^c \sum_{i,j}\frac{1}{n_y}(x_i-x_j)(x_i-x_j)^\top\\
&=
\frac{1}{2}\sum_{y=1}^c\sum_{i,j}\frac{1}{n_y}(x_ix_i^\top -x_ix_j^\top -x_jx_i^\top +x_jx_j^\top)\\
&=
\sum_{y=1}^c \left(\sum_i x_ix_i^\top - \frac{1}{n_y} \sum_{i,j}x_ix_j^\top\right).
\end{align}
よって、
\begin{align}
S\ovparen{w}  = \frac{1}{2}\sum_{i,j=1}^n Q_{i,j}\ovparen{w}(x_i-x_j)(x_i-x_j)^\top.
\end{align}
前半は示された。

後半を示す。
\begin{align}
\frac{1}{2}\sum_{i,j=1}^n Q_{i,j}\ovparen{b} (x_i-x_j)(x_i-x_j)^\top
&=
\frac{1}{2}\sum_{i,j=1}^n (1/n-Q_{i,j}\ovparen{b}) (x_i-x_j)(x_i-x_j)^\top\\
&=
\frac{1}{2n}\sum_{i,j=1}^n (x_i-x_j)(x_i-x_j)^\top - S\ovparen{w}\\
&=
\sum_{i=1}^n x_ix_i^\top - S\ovparen{w}\\
&=
S\ovparen{b}.
\end{align}
示された。


\section*{付録1}
\subsection*{学習プログラム}
\tiny
\begin{verbatim}
#include <iostream>
#include <Eigen/Dense>
#include <Eigen/Eigenvalues>
#include <cmath>
#include <random>
#include <string>
#include <fstream>
// #define M_PI 3.1416

#define print(var)  \
  std::cout<<#var"= "<<std::endl<<(var)<<std::endl

using Eigen::MatrixXd;
using Eigen::MatrixXf;
using Eigen::GeneralizedEigenSolver;

void save_mat(std::string name, const MatrixXd* mat){
  std::ofstream file(name);
  file << *mat << std::endl << std::endl;
}

float gk(const MatrixXd& x, const MatrixXd& c){
  float hh = std::pow(2*1, 2); 
  auto d = (x - c).norm();
  return std::exp(-d*d/hh);
}


int main()
{
  std::mt19937 engine;
  std::normal_distribution<> dist(0, 1);
  std::uniform_real_distribution<> unif(0, 1);

  /* make points */
  MatrixXd matX(2, 100);
  MatrixXd sumX(2, 1);
  MatrixXd vecMu(2, 2);
  // for(int i=0; i < 100; i++){
  //   matX(0, i) = dist(engine);
  //   if(i < 50){
  //     matX(0, i) -= 4;
  //   }else{
  //     matX(0, i) += 4;
  //   }
  //   matX(1, i) = dist(engine);
  // }
  for(int i=0; i < 100; i++){
    matX(0, i) = dist(engine);
    if(i < 25){
      matX(0, i) -= 4;
    }else if(25 <= i && i < 50){
      matX(0, i) += 4;
    }
    matX(1, i) = dist(engine);
  }

  /* centralize */
  for(int i=0; i < 100; i++){
    sumX += matX.col(i);
  }
  for(int i=0; i < 100; i++){
    matX.col(i) -= sumX/100;
  }

  vecMu(0, 0) = 0;
  vecMu(0, 1) = 0;
  vecMu(1, 0) = 0;
  vecMu(1, 1) = 0;
  std::cout << vecMu << std::endl;
  for(int i=0; i < 100; i++){
    if(i < 50){
      vecMu(0, 0) += matX(0, i);
      vecMu(1, 0) += matX(1, i);
    }else{
      vecMu(0, 1) += matX(0, i);
      vecMu(1, 1) += matX(1, i);
    }
    std::cout << i << "," << vecMu << std::endl;
  }
  // vecMu /= 50;

  MatrixXd matSw(2, 2);
  // std::cout << matSw << std::endl;
  for(int i=0; i < 100; i++){
    if(i < 50){
      matSw += (matX.col(i) - vecMu.col(0))*(matX.col(i) - vecMu.col(0)).transpose();
    }else{
      matSw += (matX.col(i) - vecMu.col(1))*(matX.col(i) - vecMu.col(1)).transpose();
    }
      // std::cout << i << "," << matSw << std::endl;
  }

  MatrixXd matSb(2, 2);
  matSb = 50*vecMu.col(0)*vecMu.col(0).transpose() + 50*vecMu.col(1)*vecMu.col(1).transpose();





  // MatrixXd matW(100, 100);
  // for(int i=0; i < 100; i++){
  //   for(int j=0; j < 100; j++){
  //     matW(i, j) = gk(matX.col(i), matX.col(j));
  //   }
  // }

  // MatrixXd matD(100, 100);
  // for(int i=0; i < 100; i++){
  //   matD(i, i) = matW.col(i).sum();
  // }

  // MatrixXd matL = matD - matW;

  GeneralizedEigenSolver<MatrixXd> ges;
  ges.compute(matSb, matSw, true);
  MatrixXd matTlpp(2, 2); //= ges.eigenvectors().alpha();
  for(int i=0; i < 2; i++){
    for(int j=0; j < 2; j++){
      matTlpp(i, j) = ges.eigenvectors()(i, j).real();
    }
  }
  MatrixXd bvec(2, 1);
  if(ges.eigenvalues()(0).real() > ges.eigenvalues()(1).real()){
    bvec = matTlpp.col(0);
  }else{
    bvec = matTlpp.col(1);
  }

  // MatrixXd matRed = matTlpp*matX;

  // // out
  // std::ofstream file("matTlpp");
  // file << matTlpp << std::endl;
  // std::ofstream filew("matW");
  // filew << matW << std::endl;
  // std::ofstream filed("matD");
  // filed << matD << std::endl;
  // std::ofstream filep("points");
  // filep << matX << std::endl;
  // std::ofstream fileeig("eig");
  // fileeig << ges.eigenvalues() << std::endl << ges.eigenvectors() << std::endl;
  // std::ofstream filered("matRed");
  // filered << matRed << std::endl;
  // std::ofstream filewvec("matX");
  // filewvec << matX << std::endl;
  // std::ofstream filebvec("bestVec");
  // filebvec << bvec << std::endl;
  save_mat(std::string("matSb"), &matSb);
  save_mat(std::string("matSw"), &matSw);
  save_mat(std::string("vecMu"), &vecMu);
  save_mat(std::string("matTlpp"), &matTlpp);
  save_mat(std::string("matX"), &matX);
  save_mat(std::string("bVec"), &bvec);
}
\end{verbatim}
\normalsize
\subsection*{表示プログラム}
\tiny
\begin{verbatim}
import numpy as np
import matplotlib.pyplot as plt

wvec = np.loadtxt("bVec")

matx = np.loadtxt("matX")

plt.axis([-5, 5, -5, 5])
lsp = np.linspace(-20, 20, 100)
plt.plot(matx[0, :50], matx[1, :50], 'rx')    
plt.plot(matx[0, 50:], matx[1, 50:], 'bx')    
plt.plot(lsp, lsp*(wvec[1]/wvec[0]), 'k-')  
plt.show()
\end{verbatim}

\normalsize


\end{document}
