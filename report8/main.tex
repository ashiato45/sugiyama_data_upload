\documentclass[9pt]{ltjsarticle}
\DeclareSymbolFont{bbold}{U}{bbold}{m}{n}
\DeclareSymbolFontAlphabet{\mathbbold}{bbold}
\newcommand{\bbold}{\mathbbold}
\usepackage{xcolor}
\usepackage{amsmath,amsfonts,amssymb}
\usepackage{enumitem}
\usepackage{ashiato45}
%\usepackage{okumacro}
\def\MARU#1{\textcircled{\scriptsize #1}}
\usepackage{graphicx}
\usepackage{ulem}
\usepackage{framed}
\usepackage{algorithm}
\usepackage{algorithmic}
\usepackage{here}
%\usepackage[twoside]{geometry}
\usepackage{mytheorems}
\usepackage{tikz}
% usepackage{ascmac}
% \usepackage{stmaryrd}
% \usepackage{txfonts}
% \usetikzlibrary{cd}
\title{先端データ解析論(杉山将先生・本多淳也先生)\\第8回レポート}
\author{ashiato45}
\date{2017年6月8日}

\renewcommand{\bf}{\mathbf}
\newcommand{\nemui}{Y=眠}
\newcommand{\nemukunai}{Y=非眠}
\newcommand{\suki}{X=好}
\newcommand{\kirai}{X=嫌}
\newcommand{\argmin}{\mathop{\mathrm{argmin}}}
\newcommand{\argmax}{\mathop{\mathrm{argmax}}}

\begin{document}
\maketitle

\section*{宿題1}
まず、$1-\mu^\top \phi(x)y \ge 0$のときを考える。
\begin{align}
\frac{\pd J(\mu, \Sigma)}{\pd \mu} = 
2(-\phi(x)y)(1-\hat{\mu}^\top \phi(x)y) + 2\gamma \hat\Sigma\inv (\hat \mu - \tilde \mu)
\end{align}
これが0になるときを考えると、($y^2=1$に注意して)
\begin{align}
\hat \mu = \left(\frac{\phi(x)\phi(x)^\top}{\gamma} + \hat \Sigma\inv\right)\inv\left(\frac{\phi(x)y}{\gamma} + \hat\Sigma\inv \tilde \mu\right)
\end{align}
となる。逆行列の公式より、
\begin{align}
\hat \mu &= 
\left(\hat\Sigma - \frac{\hat\Sigma\phi(x)\phi(x)^\top \hat\Sigma}{\phi(x)^\top \hat\Sigma\phi(x)+\gamma}\right)
\left(\frac{\phi(x)y}{\gamma} + \hat\Sigma\inv \tilde \mu\right)\\
&=
\left(\hat\Sigma - \frac{\hat\Sigma\phi(x)\phi(x)^\top \hat\Sigma}{\phi(x)^\top \hat\Sigma\phi(x)+\gamma}\right)\frac{\phi(x)y}{\gamma}
+
\left(\hat\Sigma - \frac{\hat\Sigma\phi(x)\phi(x)^\top \hat\Sigma}{\phi(x)^\top \hat\Sigma\phi(x)+\gamma}\right)\hat\Sigma\inv \tilde\mu\\
&=
\tilde\mu + \frac{\gamma\hat\Sigma}{\phi(x)^\top \hat\Sigma\phi(x)+\gamma}\frac{\phi(x)y}{\gamma} 
- \frac{\hat\Sigma\phi(x)\phi(x)^\top}{\phi(x)^\top \hat\Sigma\phi(x)+\gamma}\tilde \mu\\
&=
\tilde \mu + 
\frac{y-\tilde\mu^\top\phi(x)}{\phi(x)^\top \hat\Sigma\phi(x)+\gamma}\hat\Sigma\phi(x)
\end{align}
となる。

次に、$1-\mu^\top \phi(x)y<0$のときを考える。
\begin{align}
\frac{\pd J(\mu, \Sigma)}{\pd \mu} = 
2\gamma \hat\Sigma\inv (\hat \mu - \tilde \mu)
\end{align}
これが0になるときを考えると、$\hat\mu = \tilde \mu$となる。

以上の2つの場合をまとめると、
\begin{align}
\hat\mu = 
\tilde\mu + \frac{y\max(0, 1-\tilde\mu^\top \phi(x)y)}{\phi(x)^\top \hat\Sigma\phi(x)+\gamma}\hat\Sigma\phi(x)
\end{align}
を得る。

\section*{宿題2}
Python実装は付録にある。結果、図1を得た。

\easypicture{figure_1.pdf}


\section*{付録}
\tiny
\begin{verbatim}
import sys
import numpy as np
import numpy.matlib
import matplotlib.pyplot as plt
import scipy.io
import gc
import random

np.random.seed(42)

n = 50

# Prepare data
y = np.ones((n//2, 1))*np.array([1,-1])
y = y.reshape(y.size, 1).transpose()
y = np.ones((1, n))
y[0, n//2 + 1:] *= -1
# y = [1,..,1,-1,...,-1]
print(y.shape)
print(y)

x = np.random.randn(n, 2)
x[n//2 + 1:, 0] *= -1
x[0:n//2, 0] -= 15
x[n//2 + 1:, 0] -= 5
x[0:2, 0] += 10
x = np.hstack((x, np.ones((n, 1))))
x = x.transpose() # 3xn
print(x.shape)

mu = np.zeros((3, 1))
sigma = np.eye(3)
gamma = 0.5
indices = np.arange(n)
np.random.shuffle(indices)
for i in indices:
    beta = x[:, i].reshape(1, 3).dot(sigma).dot(x[:, i].reshape(3, 1)) + gamma
    mu += y[0, i]*np.maximum(0, 1-mu.transpose().dot(x[:, i])*y[0, i])/beta*sigma.dot(x[:, i].reshape(3, 1)).reshape(3, 1)
    print(sigma.dot(x[:, i]).reshape(3, 1).dot(x[:, i].reshape(1, 3)).shape)
    sigma -= sigma.dot(x[:, i]).reshape(3, 1).dot(x[:, i].reshape(1, 3)).dot(sigma)/beta

# print
plt.axis([-20, 0, -2.1, 2.1])
plt.plot(x[0, 0:n//2], x[1, 0:n//2], 'bo')   
plt.plot(x[0, n//2 + 1:], x[1, n//2 + 1:], 'rx')   
dots = np.linspace(-22, 2, 100)
plt.plot(dots, -(dots*mu[0] + mu[2])/mu[1], '-')
plt.show()
\end{verbatim}

\end{document}
