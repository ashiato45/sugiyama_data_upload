\documentclass[9pt]{ltjsarticle}
\DeclareSymbolFont{bbold}{U}{bbold}{m}{n}
\DeclareSymbolFontAlphabet{\mathbbold}{bbold}
\newcommand{\bbold}{\mathbbold}
\usepackage{xcolor}
\usepackage{amsmath,amsfonts,amssymb}
\usepackage{enumitem}
\usepackage{ashiato45}
%\usepackage{okumacro}
\def\MARU#1{\textcircled{\scriptsize #1}}
\usepackage{graphicx}
\usepackage{ulem}
\usepackage{framed}
\usepackage{algorithm}
\usepackage{algorithmic}
\usepackage{here}
%\usepackage[twoside]{geometry}
\usepackage{mytheorems}
\usepackage{tikz}
\usepackage{ascmac}
\usepackage{stmaryrd}
\usepackage{txfonts}
\usetikzlibrary{cd}
\title{先端データ解析論(杉山将先生・本多淳也先生)\\第2回レポート}
\author{ashiato45}
\date{2017年5月2日}

\renewcommand{\bf}{\mathbf}
\newcommand{\nemui}{Y=眠}
\newcommand{\nemukunai}{Y=非眠}
\newcommand{\suki}{X=好}
\newcommand{\kirai}{X=嫌}
\newcommand{\argmin}{\mathop{\mathrm{argmin}}}
\newcommand{\argmax}{\mathop{\mathrm{argmax}}}

\begin{document}
\maketitle

\section*{宿題1}
プログラムは付録に記す。
標本の分割は、標本点を5つおきに飛ばすことで5グループ作成した。
パラメタとして$h=0.2, \lambda=0.3$で実行し、
図1の結果を得た。テスト誤差の平均は、$0.668$であった。
\easypicture{figure_1.png}

\section*{付録}
\tiny
\begin{verbatim}
import sys
import numpy as np
import matplotlib.pyplot as plt

np.random.seed(42)

h = float(sys.argv[1])
lam = float(sys.argv[2])

datanum = 50
datasamples = np.linspace(-3, 3, datanum)
datasamples = datasamples.reshape(len(datasamples), 1)
y = np.sin(np.pi*datasamples)/(np.pi*datasamples) + 0.1*datasamples
y += np.random.randn(datanum, 1)*0.2 # normal distribution
# print(y)

def gk(x, c):
    return np.exp(-(x-c)**2/(2*(h**2)))

totaler = 0
for i in range(5):
    group = [j for j in range(50) if j % 5 != i]
    yy = y[group, :]
    ds = datasamples[group, :]
    phi = np.fromfunction(lambda i, j: gk(ds[i, 0], ds[j, 0]), (len(ds), len(ds)), dtype=int)

    theta = np.linalg.solve(np.dot(phi.transpose(), phi) + lam*np.eye(len(ds), dtype=int), np.dot(phi.transpose(), yy))
    graphnum = 5000
    graphsamples = np.linspace(-3, 3, graphnum)
    graphsamples = graphsamples.reshape(graphnum, 1)
    m = np.fromfunction(lambda i,j: gk(graphsamples[i, 0], ds[j, 0]), (len(graphsamples), len(ds)), dtype=int)
    ans = np.dot(m, theta)
    plt.plot(graphsamples, ans, '-')
    em = np.fromfunction(lambda i,j: gk(ds[i, 0], ds[j, 0]), (len(ds), len(ds)), dtype=int)
    eans = np.dot(em, theta)
    er = (np.linalg.norm(eans - yy)**2)
    print("error" + str(i) + ":" + str(er))
    totaler += er
    

print(theta)
print(np.sum(abs(theta) < 1e-8))
print(totaler/5.0)

plt.axis([-2.8, 2.8, -1, 1.5])    

plt.plot(datasamples, y, 'o')
# plt.plot(datasamples, eout, 'o')
plt.show()
\end{verbatim}

\end{document}
