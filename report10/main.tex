\documentclass[9pt]{ltjsarticle}
\DeclareSymbolFont{bbold}{U}{bbold}{m}{n}
\DeclareSymbolFontAlphabet{\mathbbold}{bbold}
\newcommand{\bbold}{\mathbbold}
\usepackage{xcolor}
\usepackage{amsmath,amsfonts,amssymb}
\usepackage{enumitem}
\usepackage{ashiato45}
%\usepackage{okumacro}
\def\MARU#1{\textcircled{\scriptsize #1}}
\usepackage{graphicx}
\usepackage{ulem}
\usepackage{framed}
\usepackage{algorithm}
\usepackage{algorithmic}
\usepackage{here}
%\usepackage[twoside]{geometry}
\usepackage{mytheorems}
\usepackage{tikz}
% usepackage{ascmac}
% \usepackage{stmaryrd}
% \usepackage{txfonts}
% \usetikzlibrary{cd}
\title{先端データ解析論(杉山将先生・本多淳也先生)\\第10回レポート}
\author{ashiato45}
\date{2017年6月20日}

\renewcommand{\bf}{\mathbf}
\newcommand{\nemui}{Y=眠}
\newcommand{\nemukunai}{Y=非眠}
\newcommand{\suki}{X=好}
\newcommand{\kirai}{X=嫌}
\newcommand{\argmin}{\mathop{\mathrm{argmin}}}
\newcommand{\argmax}{\mathop{\mathrm{argmax}}}
\newcommand{\test}{\mathrm{test}}
\newcommand{\train}{\mathrm{train}}
\newcommand{\kitai}{\bbold{E}}
\newcommand{\tr}{\mathrm{tr}}

\begin{document}
\maketitle

\section*{宿題1}
\begin{align}
\sum_{i,i'} W_{i,i'}\norm{Tx_i-T{x_{i'}}}^2
&=
\sum_{i,i'} W_{i,i'}(Tx_i-T{x_{i'}})^\top (Tx_i-T{x_{i'}})\\
&=
\sum_{i,i'} W_{i,i'}(x_i^\top T^\top T x_i - x_i^\top T^\top T x_{i'} - x_{i'}^\top T^\top T x_i + x_{i'}^\top T^\top T x_{i'})\\
&=
2\left({\sum_{i,i'}W_{i,i'}x_i^\top T^\top T x_i} 
- {\sum_{i,i'}W_{i,i'}x_i^\top T^\top T x_{i'}}\right)\\
&=
2\left(\underbrace{\sum_{i,i',a,b,c}W_{i,i'} X_{ai}T_{ba}T_{bc}X_{ci}}_{\alpha \defeq} 
- \underbrace{\sum_{i,i'a,b,c}W_{i,i'}X_{ai}T_{ba}T_{bc}X_{ci'}}_{\beta \defeq}\right)\\
\end{align}

\begin{align}
(TXDX^\top T^\top)_{ij}
&=
\sum_{a,b,c,d}T_{ia}X_{ab}D_{bc}(X^\top)_{cd}(T^\top)_{dj}\\
&=
\sum_{a,b,c,d}T_{ia}X_{ab}D_{bc}X_{dc}T_{jd}\\
&=
\sum_{a,b,d}T_{ia}X_{ab}(\sum_{i'}W_{b,i'})X_{db}T_{jd}.
\end{align}
よって、
\begin{align}
\tr(TXDX^\top T^\top)
&=
\sum_{i,i'a,b,d}T_{ia}X_{ab}W_{b,i'}X_{db}T_{id}\\
&= \alpha.
\end{align}

\begin{align}
(TXWX^\top T^\top)_{ij}
&=
\sum_{a,b,c,d}T_{ia}X_{ab}W_{bc}(X^\top)_{cd}(T^\top)_{dj}\\
&=
\sum_{a,b,c,d}T_{ia}X_{ab}W_{bc}X_{dc}T_{jd}.
\end{align}
よって、
\begin{align}
\tr(TXWX^\top T^\top)
&=
\sum_{a,b,c,d,i}T_{ia}X_{ab}W_{bc}X_{dc}T_{id}\\
&=
\beta.
\end{align}

よって、
\begin{align}
\sum_{i,i'} W_{i,i'}\norm{Tx_i-T{x_{i'}}}^2
&=
2\left(\tr(TXDX^\top T^\top) - \tr(TXWX^\top T^\top)\right)\\
&=
2\left(\tr(TX(D-W)X^\top T^\top)\right)\\
&=
2\left(\tr(TXLX^\top T^\top)\right).
\end{align}


\section*{宿題2}
C++とEigenによる実装(表示にはPythonを用いた)は付録1にある。結果、図1と図2を得た。
  \easypicture{figure_1.pdf}
  \easypicture{figure_2.pdf}

\section*{付録1}
\subsection*{学習プログラム}
\tiny
\begin{verbatim}
#include <iostream>
#include <Eigen/Dense>
#include <Eigen/Eigenvalues>
#include <cmath>
#include <random>
#include <string>
#include <fstream>
#define M_PI 3.1416

#define print(var)  \
  std::cout<<#var"= "<<std::endl<<(var)<<std::endl

using Eigen::MatrixXd;
using Eigen::MatrixXf;
using Eigen::GeneralizedEigenSolver;

void save_mat(std::string& name, const MatrixXd* mat){
  std::ofstream file(name);
  file << *mat << std::endl;
}

float gk(const MatrixXd& x, const MatrixXd& c){
  float hh = std::pow(2*1, 2); 
  auto d = (x - c).norm();
  return std::exp(-d*d/hh);
}


int main()
{
  std::mt19937 engine;
  std::normal_distribution<> dist(0, 1);
  std::uniform_real_distribution<> unif(0, 1);

  // MatrixXd matX(2, 100);
  // for(int i=0; i < 100; i++){
  //   matX(0, i) = 2*dist(engine);
  //   matX(1, i) = dist(engine);
  // }
  MatrixXd matX(2, 100);
  for(int i=0; i < 100; i++){
    matX(0, i) = dist(engine);
    matX(1, i) = 2*std::round(unif(engine)) - 1 + dist(engine)/3;
  }

  MatrixXd matW(100, 100);
  for(int i=0; i < 100; i++){
    for(int j=0; j < 100; j++){
      matW(i, j) = gk(matX.col(i), matX.col(j));
    }
  }

  MatrixXd matD(100, 100);
  for(int i=0; i < 100; i++){
    matD(i, i) = matW.col(i).sum();
  }

  MatrixXd matL = matD - matW;

  GeneralizedEigenSolver<MatrixXd> ges;
  ges.compute(matX*matL*matX.transpose(), matX*matD*matX.transpose(), true);
  MatrixXd matTlpp(2, 2); //= ges.eigenvectors().alpha();
  for(int i=0; i < 2; i++){
    for(int j=0; j < 2; j++){
      matTlpp(i, j) = ges.eigenvectors()(i, j).real();
    }
  }
  MatrixXd wvec(2, 1);
  if(ges.eigenvalues()(0).real() < ges.eigenvalues()(1).real()){
    wvec = matTlpp.col(0);
  }else{
    wvec = matTlpp.col(1);
  }

  MatrixXd matRed = matTlpp*matX;

  // out
  std::ofstream file("matTlpp");
  file << matTlpp << std::endl;
  std::ofstream filew("matW");
  filew << matW << std::endl;
  std::ofstream filed("matD");
  filed << matD << std::endl;
  std::ofstream filep("points");
  filep << matX << std::endl;
  std::ofstream fileeig("eig");
  fileeig << ges.eigenvalues() << std::endl << ges.eigenvectors() << std::endl;
  std::ofstream filered("matRed");
  filered << matRed << std::endl;
  std::ofstream filewvec("worstVec");
  filewvec << wvec << std::endl;
}
\end{verbatim}
\normalsize
\subsection*{表示プログラム}
\tiny
\begin{verbatim}
import numpy as np
import matplotlib.pyplot as plt

wvec = np.loadtxt("worstVec")

matx = np.loadtxt("points")

plt.axis([-5, 5, -5, 5])
lsp = np.linspace(-20, 20, 100)
plt.plot(matx[0, :], matx[1, :], 'rx')     
plt.plot(lsp, lsp*(wvec[1]/wvec[0]), '-')  
plt.show()
\end{verbatim}

\normalsize


\end{document}
