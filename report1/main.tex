\documentclass[9pt]{ltjsarticle}
\DeclareSymbolFont{bbold}{U}{bbold}{m}{n}
\DeclareSymbolFontAlphabet{\mathbbold}{bbold}
\newcommand{\bbold}{\mathbbold}
\usepackage{xcolor}
\usepackage{amsmath,amsfonts,amssymb}
\usepackage{enumitem}
\usepackage{ashiato45}
%\usepackage{okumacro}
\def\MARU#1{\textcircled{\scriptsize #1}}
\usepackage{graphicx}
\usepackage{ulem}
\usepackage{framed}
\usepackage{algorithm}
\usepackage{algorithmic}
\usepackage{here}
%\usepackage[twoside]{geometry}
\usepackage{mytheorems}
\usepackage{tikz}
\usepackage{ascmac}
\usepackage{stmaryrd}
\usetikzlibrary{cd}
\title{先端データ解析論(杉山将先生・本多淳也先生)}
\author{ashiato45}
\date{2017年4月16日}

\renewcommand{\bf}{\mathbf}
\newcommand{\nemui}{Y=眠}
\newcommand{\nemukunai}{Y=非眠}
\newcommand{\suki}{X=好}
\newcommand{\kirai}{X=嫌}

\begin{document}
\maketitle

\section*{宿題1}
\begin{enumerate}[label=(\Alph*)]
 \item $p(\suki, \nemui)=p(\nemui | \suki)p(\suki)=0.25\cdot 0.8 = 0.2 .$
 \item 
\begin{align}
 p(\nemui)
&=
p(\suki, \nemui) + p(\kirai, \nemui)\\
 & =
p(\nemui | \suki)p(\suki) + p(\nemui | \kirai)p(\kirai)\\
 & =
0.25\cdot 0.8 + 0.25\cdot 0.2\\
 & =
0.25.
\end{align}
 \item 
$p(\suki | \nemui) = p(\suki, \nemui)/p(\nemui) = 0.2/0.25 = 0.8.$
 \item 
(A)によれば、$p(\suki, \nemui)=0.2$である。一方、(B)によれば、
$p(\suki)p(\nemui)=0.8\cdot 0.25 = 0.2$である。よって、
$p(\suki, \nemui)=p(\suki)p(\nemui)$であり、統計の好き嫌いと授業中眠たい
       ことは独立である。
\end{enumerate}

\section*{宿題2}
測度と確率変数とをうまくとれば連続型は離散型を含むので、連続型のときのみ示す。積分
の線形性から従う。
\begin{enumerate}[label=(\Alph*)]
 \item $E(c)=\int cdx = c\int dx = c\cdot 1 = c.$
 \item $E(X+X')=\int(X+X')dx = \int Xdx + \int X'dx = E(X)+E(X').$
 \item $E(cX)=\int(cX)dx = c\int Xdx = cE(X).$ 
\end{enumerate}

\section*{宿題3}
\begin{enumerate}[label=(\Alph*)]
 \item $V(c)=E[(c-E[c])^2]=E[0]=0.$
 \item $V[X+c]=E[(X+c)^2]-E[X+c]^2=(E[X^2]+2cE[X]+c^2)-(E[X]^2+2cE[X]+c^2)=E[X^2]-E[X]^2=V[X].$
 \item $V[cX]=E[(cX)^2]-E[cX]^2=c^2(E[X^2]-E[X]^2)=c^2V[X].$
 \item 
\begin{align}
 V(X+X')
 &= E[(X+X')^2]-E[X+X']^2\\
 & = (E[X^2]-E[X]^2) + (E[{X'}^2]-E[X']^2) + 2(E[XX']-E[X]E[X'])\\
 & = V[X] + V[X'] + 2\mathrm{Cov}(X,X').
\end{align}
\end{enumerate}
\end{document}
